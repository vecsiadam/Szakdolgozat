% !TEX encoding = UTF-8 Unicode

\documentclass[a4paper,12pt]{article}

% Set margins
\usepackage[hmargin=3cm, vmargin=3cm]{geometry}

\frenchspacing

% Language packages
\usepackage[utf8]{inputenc}
\usepackage[T1]{fontenc}
\usepackage[magyar]{babel}

% AMS
\usepackage{amssymb,amsmath}

% Graphic packages
\usepackage{graphicx}

% Colors
\usepackage{color}
\usepackage[usenames,dvipsnames]{xcolor}

% Enumeration
\usepackage{enumitem}

% Links
\usepackage{hyperref}

\linespread{1.2}

\begin{document}

\pagestyle{empty}

\section*{Summary}

The summary  presented the topic of artistic filters with their mathematical model and some of their own filtering implementations.\\

I have never been interested in with image processing or algorithms that I've mentioned and used here until now. I was always interested  how this works. In the first couple of months I did several background research about algorithms and their mathematical background.\\

As you can see in the previous chapters, I made 4 filters which are cartoon, pencil, and painting  like.I presented my mathematics and its implementation. Some of these have been written using sources, but they were based on my own idea. The mathematical background of filter algorithms was no longer unknown after the previous background research.  So I only needed a tool to implement.\\

I never used the OpenCV library before. The C ++ programming language was the most convenient, initially coded in C but there are several such algorithm implementations missing from the OpenCV library, which can be found in C ++, but these algorithms are important for self-made filters. Algorithms can be easily used in the directory such as Chapter 5. Simply enter the desired parameters and have reached the desired operation.\\

Tests have been used to determine the calculation duration of each step is used during using the filters. This makes it more visible for cost-cutting operations. Videos and real-time image processing in some of the cases, the self-made filters in some places are not working properly. The video image vibrates like Chapter 6. as I mentioned it, but it's also visible on the CD attached to the exam if we run these applications. I have proposed an example correction to them, which is  only at the theoretical level .It might be time to devise this.

\end{document}
