% !TEX encoding = UTF-8 Unicode

\Chapter{Implementáció C++ és OpenCV segítségével}
\Section{OpenCV bemutatása}

Az OpenCV (Open Source Computer Vision Library) egy nyílt forráskódú számítógépes látás és gépi tanulási szoftverkönyvtár. Az OpenCV-t azért hozták létre, hogy közös infrastruktúrát biztosítson a számítógépes megjelenítési alkalmazások számára, és felgyorsítsa a gépi érzékelés használatát a kereskedelmi termékekben. 
\\

\noindent Az OpenCV (Open Source Computer Vision Library) egy BSD licenc alatt került kiadásra, ezért ingyenes mind tudományos, mind kereskedelmi célokra. C ++, Python és Java interfészekkel rendelkezik, és támogatja a Windows, Linux, Mac OS, iOS és Android rendszereket. Az OpenCV-et a számítási hatékonyságra tervezték, és nagy hangsúlyt fektetett a valós idejű alkalmazásokra. Az optimalizált C / C ++-ban írt, a könyvtár kihasználhatja a többmagos feldolgozást is. Az OpenCL használatával kihasználhatja az alapul szolgáló heterogén számítási platform hardveres gyorsítását.
\\

\noindent A könyvtár több mint 2500 optimalizált algoritmussal rendelkezik, amely magába foglalja mind a klasszikus, mind a legmodernebb számítógépes látásmódot és a gépi tanulási algoritmusokat. Ezek az algoritmusok felismerhetik az arcokat, felismerhetik az objektumokat, osztályozhatják az emberi cselekvéseket a videókban, nyomon követhetik a mozgásokat, követhetik a mozgó objektumokat, eltávolíthatja a vörös szemeket a vakuval készített képekből, követheti a szemmozgásokat, felismerheti a tájat stb. 
\\

\Section{Szűrők implementációja OpenCV és C++-al}
\SubSection{Cartoon-style filter}
\SubSection{Pencil sketch filter}
\SubSection{Cartoon filter}
\SubSection{Paint-stlye filter}





% Technikai jelleg rész